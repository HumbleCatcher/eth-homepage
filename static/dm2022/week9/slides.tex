\documentclass[t,dvipsnames]{beamer}
\usepackage{cancel}
\usepackage{mathtools}
\graphicspath{{./assets/}}



\title{Diskrete Mathematik - Woche 8}
\author{Andreas Ellison}
\usetheme{Madrid}
\setbeamertemplate{navigation symbols}{}
\DeclareMathOperator{\isless}{less}

\DeclareMathOperator{\defn}{\stackrel{\text{def}}{\iff}}
\DeclareMathOperator{\defneq}{\stackrel{\text{def}}{\quad = \quad}}
\DeclareMathOperator{\idr}{id}
\newcommand{\isrel}[3]{#1 \, #3 \, #2}


\DeclareMathOperator{\lcm}{lcm}
\DeclareMathOperator{\ord}{ord}

\newcommand{\beamerblack}{
\setbeamercolor{frametitle}{fg=white}
\setbeamercolor{frametitle}{bg=black}
\setbeamercolor{background canvas}{bg=black}
\setbeamercolor{normal text}{fg=white}
\usebeamercolor[fg]{normal text}
\setbeamertemplate{footline}{}
\setbeamertemplate{itemize items}[circle]
\setbeamercolor{itemize item}{fg=white,bg=white}
\setbeamertemplate{enumerate items}[default]
\setbeamercolor{enumerate items}{fg=white,bg=white}
\setbeamercolor{itemize item}{fg=white,bg=white}
}

\newcounter{excounter}
\newcommand{\excount}{\stepcounter{excounter} \theexcounter{}}

\newcommand{\hintIllustration}{\textcolor{Magenta}{(Bild)}}


\begin{document}
{
\beamerblack
\begin{frame}
	\frametitle{Challenge!}
	Seien $a_1, a_2, \ldots, a_n \in \mathbb{Z}$.
	Gebe einen $\mathcal{O}(n)$ Algorithmus
	an, der eine zusammenhängende Teilfolge
	$a_i, a_{i + 1}, \ldots, a_j$ findet mit
	$1 \le i \le j \le n$, so dass
	$n \; \vert \; (a_i + a_{i + 1} + \ldots + a_j)$.

	Eine genaue Laufzeitanalyse vom angegebenem Algorithmus ist nicht
	nötig. \\~

	\textit{Tipp:} Betrachte die Summen
	$S_i \coloneqq \sum_{k=1}^{i} a_k$ für $1 \le i \le n$.
\end{frame}
}

\begin{frame}
	\frametitle{Die Eulerfunktion}
	\framesubtitle{Recap}
	Die Eulerfunktion $\varphi$ gibt uns die Anzahl Elemente in $\mathbb{Z}_n$, die teilerfremd sind zu $n$.
	\only<+(1)->{
		\begin{Lemma}
			Sei $n = \prod_{i=1}^r p_i^{e_i}$ die Primfaktorzerlegung einer Zahl $n$. Dann ist
			$$
				|\mathbb{Z}_n^*| = \varphi(n) = \prod_{i=1}^{r} (p_i - 1) p_i^{e_i - 1}.
			$$
		\end{Lemma}
		In anderen Worten gibt uns $\varphi(n)$ die Ordnung der multiplikativen Gruppe $\mathbb{Z}_n^*$.
	}
	\only<+(1)-> {

		\begin{Beispiel}
			Sei $n = 18 = 2 \cdot 3^2$. Dann ist $\varphi(n) =$ \only<+(1)>{$(2 - 1)2^0 \cdot (3 - 1)3^{1} = 6$.}
		\end{Beispiel}
	}
\end{frame}


\begin{frame}
	\frametitle{Die Eulerfunktion}
	\framesubtitle{Recap}
	Wir erinnern uns an folgendes wichtige Lemma:
	\only<+(1)->{
		\begin{Lemma}
			Sei $G$ eine endliche Gruppe mit Neutralelement $e$. Dann gilt
			für alle $a \in G$:
			$$a^{|G|} = e$$
		\end{Lemma}
	}

	\only<+(1)->{
		Daraus folgt ganz einfach:
		\begin{Lemma}
			Für alle $a \in \mathbb{Z}_n^*$:
			$$
				a^{\varphi(n)} \equiv_n 1
			$$
			\only<+(1)>{
				und für eine Primzahl $p$, für alle $1 \le a < p$:
				$$
					a^{p - 1} \equiv_p 1
				$$
			}
		\end{Lemma}
	}
\end{frame}



{
\beamerblack
\begin{frame}
	\frametitle{Aufgabe \excount}
	Let $c = 7$ be a message encrypted with the public key pair $(n, e)$. Find both the secret key $d$ and the original message $m$.
\end{frame}
}

\begin{frame}
	\frametitle{Modulare Arithmetik Tricks}
	Innerhalb von $R_n(\ldots)$ dürfen wir alles machen, was wir wollen, solange wir die Kongruenz Modulo $n$ der Terme nicht verändern, \textbf{ausser in den Exponenten.}
	\begin{itemize}[<+(1)->]
		\item $R_3(5^3) = R_3(R_3(5)^3) = R_3(2^3) = R_3(8) = 2$
		\item $R_3(5^3) \neq R_3(5^{R_3(3)}) = R_3(5^0) = 1$
	\end{itemize}
\end{frame}

{
\beamerblack
\begin{frame}
	\frametitle{Modulare Arithmetik Tricks}
	\framesubtitle{Kongruenz zu 1}
	Berechne
	$$
		R_{18}(37^{42}) = \only<+(1)->{1}
	$$

\end{frame}
}

{
\beamerblack
\begin{frame}
	\frametitle{Modulare Arithmetik Tricks}
	\framesubtitle{Der ``-'' Trick}
	Berechne
	$$
		R_3(5^{2022}) = \only<+(1)->{1}
	$$

\end{frame}
}



{
\beamerblack
\begin{frame}
	\frametitle{Modulare Arithmetik Tricks}
	\framesubtitle{Fermat's Little Theorem}
	Berechne
	$$
		R_7(1984^6) = \only<+(1)->{1}
	$$
	\textit{Hint}: $7$ teilt $1984$ nicht.

\end{frame}
}

{
\beamerblack
\begin{frame}
	\frametitle{Modulare Arithmetik Tricks}
	\framesubtitle{Fermat's Little Theorem}
	Berechne
	$$
		R_{11}(2^{1408}) = \only<+(1)->{3}
	$$

\end{frame}
}


{
\beamerblack
\begin{frame}
	\frametitle{Modulare Arithmetik Tricks}
	\framesubtitle{Fermat's Little Theorem}
	Berechne
	$$
		R_{11}(2^{3^{40}}) = \only<+(1)->{2}
	$$

\end{frame}
}

{
\beamerblack
\begin{frame}[c]
	\begin{center}
		\Huge Kahoot!
	\end{center}
\end{frame}
}

\end{document}