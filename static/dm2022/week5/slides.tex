\documentclass[t,dvipsnames]{beamer}
\usepackage{cancel}
\graphicspath{{./assets/}}



\title{Diskrete Mathematik - Woche 5}
\author{Andreas Ellison}
\usetheme{Madrid}
\setbeamertemplate{navigation symbols}{}

\DeclareMathOperator{\isless}{less}
\DeclareMathOperator{\defn}{\stackrel{\text{def}}{\iff}}
\DeclareMathOperator{\defneq}{\stackrel{\text{def}}{\quad = \quad}}
\DeclareMathOperator{\idr}{id}
\newcommand{\isrel}[3]{#1 \, #3 \, #2}

\newcommand{\beamerblack}{
\setbeamercolor{frametitle}{fg=white}
\setbeamercolor{frametitle}{bg=black}
\setbeamercolor{background canvas}{bg=black}
\setbeamercolor{normal text}{fg=white}
\usebeamercolor[fg]{normal text}
\setbeamertemplate{footline}{}
}

\newcounter{excounter}
\newcommand{\excount}{\stepcounter{excounter} \theexcounter{}}

\newcommand{\hintIllustration}{\textcolor{Magenta}{(Bild)}}


\begin{document}


\begin{frame}
	\frametitle{Relationen - Basics}
	Im folgenden seien $\rho$ und $\sigma$ Relationen auf $A$. (Also $\rho \subseteq A \times A$)
	\only<2->{
		\begin{block}{Def. Inverse}
			\begin{center}
				$\hat{\rho} \defneq{} \{(b, a) \mid (a, b) \in \rho\}$
			\end{center}
			oder alternativ, für alle $a, b \in A$
			\begin{align*}
				\isrel{b}{a}{\hat{\rho}} \iff \isrel{a}{b}{\rho}
			\end{align*}
		\end{block}
		Wir drehen also einfach alle Paare in $\rho$ um. \hintIllustration
	}
	\only<3->{
	\begin{block}{Def. Verkettung}
		\begin{align*}
			\isrel{a}{c}{\, \rho \circ \sigma \,} \defn{} \isrel{a}{b}{\rho} \land \isrel{b}{c}{\sigma} \text{ für ein $b \in A$}.
		\end{align*}
	\end{block}
	\hintIllustration.
	Wir schreiben auch $\underbrace{\rho \circ \rho \circ \ldots \circ \rho}_{\text{n Mal}} = \rho^n$.
	}
\end{frame}

\begin{frame}
	\frametitle{Relationen - Reflexivität}
	\begin{block}{Def. reflexiv}
		\begin{align*}
			\rho \text{ reflexiv } & \defn{} \forall a \in A \quad \isrel{a}{a}{\rho} \\
			                       & \iff	\forall a \in A \quad (a, a) \in \rho
		\end{align*}
	\end{block}
	Das heisst nicht anderes als $\idr \subseteq A$.
	\begin{itemize}[<+(1)->]
		\item Wie sieht die Identitätsrelation in $\mathbb{R}^2$ aus?
		\item Ist $\varnothing$ im Allgemeinen reflexiv?
		\item Ist eine nicht irreflexive Relation relfexiv? \\
		      ($\rho$ irreflexiv $\defn{} \forall a \in A \quad (a, a) \not \in \rho$)
		\item Ist $\varnothing$ im Allgemeinen irreflexiv?
	\end{itemize}
\end{frame}

\begin{frame}
	\frametitle{Relationen - Symmetrie}
	\begin{block}{Def. symmetrisch}
		$\rho$ ist symmetrisch, wenn für alle $a, b \in A$
		$$
			\isrel{a}{b}{\rho} \iff  \isrel{b}{a}{\rho}.
		$$
	\end{block}
	Man kann also $\rho$ immer umdrehen. In anderen Worten $\rho = \hat{\rho}$.
	\only<2->{
		\begin{itemize}[<+(1)->]
			\item Wie sieht Symmetrie in einem Graphen aus?
			\item Ist $\varnothing$ im Allgemeinen symmetrisch?
		\end{itemize}
	}
\end{frame}

\begin{frame}
	\frametitle{Relationen - Transitivität}
	\begin{block}{Def. transitiv}
		$\rho$ ist transitiv, wenn für alle $a, b, c \in A$ so dass
		$\isrel{a}{b}{\rho}$ und $\isrel{b}{c}{\rho}$,
		gilt auch $\isrel{a}{c}{\rho}$.

		\only<+(1)->{
			Anders gesagt soll
			$$
				\isrel{a}{b}{\rho} \land \isrel{b}{c}{\rho} \implies \isrel{a}{c}{\rho}.
			$$
			für alle $a, b, c \in A$ gelten.
		}
	\end{block}
	\only<+(1)->{
		Wenn man von $a$ nach $c$ kommen will, dann kann man die Brücke über $b$ nehmen.
	}
	\begin{itemize}[<+(1)->]
		\item Wie sieht Transitivität in einem Graphen aus?
	\end{itemize}
\end{frame}

{
\beamerblack
\begin{frame}
	\frametitle{Aufgabe \excount{}}
	Beweise:
	\begin{equation*}
		\rho \mbox{ ist transitiv} \iff \rho^2 \subseteq \rho
	\end{equation*}
\end{frame}
}

\begin{frame}
	\frametitle{Relationen - Transitivität}
	\begin{block}{Def. transitiv}
		$\rho$ ist transitiv, wenn für alle $a, b, c \in A$ so dass
		$\isrel{a}{b}{\rho}$ und $\isrel{b}{c}{\rho}$,
		gilt auch $\isrel{a}{c}{\rho}$.
	\end{block}
	Wenn man von $a$ nach $c$ kommen will, dann kann man die Brücke über $b$ nehmen.
	\begin{itemize}
		\item Wie sieht Transitivität in einem Graphen aus?
		\item<+(1)-> Ist $\varnothing$ im Allgemeinen transitiv?
	\end{itemize}
	\only<+(1)->{
		\begin{block}{Def. transitive Hülle}
			$$
				\rho^* = \bigcup_{n = 1}^{\infty} \rho^{n}
			$$
		\end{block}
		In Worten: nehme $\rho$ und mache ihn transitiv. Drückt in einem Graphen Erreichbarkeit aus. \hintIllustration
	}
\end{frame}

{
\beamerblack
\begin{frame}
	\frametitle{Challenge!}
	Beweise:
	\begin{equation*}
		\rho \mbox{ transitiv } \implies \rho^* = \rho.
	\end{equation*}
\end{frame}
}

{
\beamerblack
\begin{frame}
	\frametitle{Aufgabe \excount{} (Alte Prüfungsaufgabe FS21)}
	Sei
	$$\rho = \{(1, 2), (2, 3), (3, 4), (4, 2), (4, 4)\}$$
	eine Relation auf $\{1, 2, 3, 4\}$. \\~

	Bestimme $\rho^3$. (Lösung in Notizen)
\end{frame}
}

\begin{frame}
	\frametitle{Äquivalenzrelationen}
	Nehmen wir all diese Eigenschaften zusammen, so kommen wir auf ein einfaches aber wichtiges Konzept:
	\only<+(1)->{
		\begin{block}{Def. Äquivalenzrelation}
			Wir nennen $\rho$ eine Äquivalenzrelation, wenn $\rho$ reflexiv, symmetrisch und transitiv ist.
		\end{block}

	}

	\only<+(1)->{
		Nicht überraschenderweise geht es da darum, die Äquivalenz zwischen Objekten aufzufassen; also dass wir Objekte in einem gewissen Sinne als ``gleich" betrachten können. Jedes der drei Eigenschaften fasst Merkmale auf, die wir intuitiv von äquivalenten Objekten erwarten würden. Einfaches Beispiel ist Gleichheit (``$=$").
	}

	\only<+(1)->{
		\begin{itemize}
			\item Ist $\idr$ eine Äquivalenzrelation?
		\end{itemize}
	}
\end{frame}

\begin{frame}
	\frametitle{Äquivalenzklassen}
	Elemente, die untereinander durch die Relation verknüpft sind, bilden Gruppen \hintIllustration.
	\only<2->{
		Im folgenden sei $\sim$ eine Äquivalenzrelation.
		Diese Gruppen (Äquivalenzklassen) wollen wir irgendwie beschreiben können.
		\begin{block}{Def. Äquivalenzklasse}
			Sei $a \in A$. Die Menge aller Elemente, die unter $\sim$ zu $a$ äquivalent sind schreiben wir als
			$$
				[a]_\sim \defneq{} \{b \in A \mid \isrel{b}{a}{\sim}\}
			$$.
		\end{block}
	}
	\only<3->{
		Reflexivität erfordert, dass jedes Element $a \in A$ irgendwie in der Relation ist. Daher ist intuitiv klar, dass
		$\sim$ die ganze Menge $A$ in diese Gruppen aufteilt. Die Menge der Äquivalenzklassen $A / \sim$ ist also eine Partition von $A$. (Das muss man aber beweisen!)
	}
\end{frame}

{
\beamerblack
\begin{frame}
	\frametitle{Aufgabe \excount{}  (Aufgabe vom Skript)}
	Prove that the intersection of two equivalence relations is an equivalence relation.
\end{frame}
}

{
\beamerblack
\begin{frame}
	\frametitle{Aufgabe \excount{}  (Alte Prüfungsaufgabe HS21)}
	Let $\rho$ and $\sigma$ be two equivalence relations on a set $A$.
	Prove that if
	$$\rho \circ \sigma = \sigma \circ \rho$$
	then $\rho \circ \sigma$ is an
	equivalence relation.
\end{frame}
}

\begin{frame}
	\frametitle{Partielle Ordnungen}
	Eine Eigenschaft von Relationen fehlt uns noch:
	\begin{block}{Def. Antisymmetrie}
		$\rho$ ist antisymmetrisch, wenn für alle $a, b \in A$ folgende Implikation gilt
		$$
			\isrel{a}{b}{\rho} \land \isrel{b}{a}{\rho} \implies  a = b.
		$$
	\end{block}
	\only<+(1)->{
		Dann haben wir alles für eine weiteres wichtiges Konzept. \hintIllustration
		\begin{block}{Def. partielle Ordnung}
			Wir nennen $\preceq$ eine partielle Ordnung, wenn $\preceq$ reflexiv, \textbf{anti}symmetrisch und transitiv ist.
		\end{block}
		Musterbeispiel: ``$\le$''.
	}
\end{frame}


\begin{frame}
	\frametitle{Hasse Diagramme}
	(Endliche) partielle Ordnungen können wir bildlich darstellen.
	\begin{Beispiel}
		Zeichne das Hasse Diagramm der partiellen Ordnung $(\{2, 4, 6, 12, 18\}; |)$.
	\end{Beispiel}
	\only<+(1)->{
		(Werdet ihr nicht viel benutzen und es kam in den letzten Jahren selten dran. Definitionen wie ``minimal element'', ``least element'' würde ich auch einfach schnell auf den Spick schreiben.)
	}

	\only<+(1)->{
		Folgende Definition ist aber wichtig zu kennen:
		\begin{block}{Def. totale Ordnung}
			Wir nennen $(A; \preceq)$ eine totale Ordnung, falls jede zwei Elemente \textit{vergleichbar} (\textit{comparable}) sind.
			Also für jedes $a, b \in A$ gilt entweder $a \preceq b$ oder $b \preceq a$.
		\end{block}
		\begin{itemize}[<+(1)->]
			\item Wie sieht das Hasse Diagramm einer partiellen Ordnung aus?
		\end{itemize}
	}
\end{frame}


\begin{frame}
	\frametitle{Hasse Diagramme}
	\includegraphics[width=\textwidth]{total-order-java-1}
	\includegraphics[width=\textwidth]{total-order-java-2}
\end{frame}
\end{document}