\documentclass[t,dvipsnames]{beamer}
\usepackage{cancel}
\usepackage{mathtools}
\graphicspath{{./assets/}}



\title{Diskrete Mathematik - Woche 13}
\author{Andreas Ellison}
\usetheme{Madrid}
\setbeamertemplate{navigation symbols}{}

\newcommand{\beamerblack}{
\setbeamercolor{frametitle}{fg=white}
\setbeamercolor{frametitle}{bg=black}
\setbeamercolor{background canvas}{bg=black}
\setbeamercolor{normal text}{fg=white}
\usebeamercolor[fg]{normal text}
\setbeamertemplate{footline}{}
\setbeamertemplate{itemize items}[circle]
\setbeamercolor{itemize item}{fg=white,bg=white}
\setbeamertemplate{enumerate items}[default]
\setbeamercolor{enumerate items}{fg=white,bg=white}
\setbeamercolor{itemize item}{fg=white,bg=white}
}

\newcounter{excounter}
\newcommand{\excount}{\stepcounter{excounter} \theexcounter{}}

\newcommand{\hintIllustration}{\textcolor{Magenta}{(Bild)}}
\newcommand{\deriv}[1]{\vdash_{#1}}


\begin{document}

\begin{frame}
	\frametitle{Prenex normal form}
	\begin{Lemma}[6.8.]
		Seien $F, H$ Formeln, so dass $x$ nicht frei in $H$ vorkommt. Dann gilt
		\includegraphics[width=0.7\textwidth]{lemma6-8.png}
	\end{Lemma}
\end{frame}

{
\beamerblack
\begin{frame}
	\frametitle{Prenex normal form (FS19)}
	For the formula
	$$
		P(x, x) \land \forall x \, ((\exists y \, P(x, y)) \rightarrow Q(z)),
	$$
	give an equivalent formula in the prenex normal form. \\~


	\only<+(1)->{
		Vorgehen:
		\begin{enumerate}[<+(1)->]
			\item Bounded Variablen umbenennen.
			\item Quantoren rausziehen (mithilfe von Lemma 6.8.).
		\end{enumerate}
	}
\end{frame}
}

\begin{frame}
	\frametitle{Die zwei wichtigsten Facts zum Resolutionskalkül}
	\only<+(1)->{
		\begin{lemma}[6.6.]
			Die Regel $\deriv{\mathrm{Res}}$ ist korrekt.
		\end{lemma}
	}

	\only<+(1)->{
		\begin{theorem}[6.7.]
			$F$ ist unerfüllbar $\iff F \deriv{\mathrm{Res}} \varnothing$

			(Angenommen $F$ in KNF.)
		\end{theorem}
	}
\end{frame}

{
\beamerblack
\begin{frame}
	\frametitle{Resolution calculus (FS22)}
	Use the resolution calculus to prove that $A \land C$ is a logical consequence of
	$$
		M = \{\neg B \lor A, \neg A \rightarrow B, A \rightarrow C\}.
	$$
	\only<+(1)->{
		Vorgehen:
		\begin{enumerate}[<+(1)->]
			\item Formel $F'$ in KNF finden, so dass $F'$ unerfüllbar $\iff$ Aussage erfüllt.
			\item Resolutionskalkül auf $F'$ anwenden.
		\end{enumerate}
	}
\end{frame}
}

{
\beamerblack
\begin{frame}
	\frametitle{Resolution calculus (FS20)}
	Sei $F = ((A \lor B) \land (A \rightarrow C) \land (B \rightarrow C)) \rightarrow C$. Wir wollen mit dem Resolutionskalkül beweisen, dass $F$ eine Tautologie ist. Welche Formel $F'$ in KNF können wir dafür wählen?
\end{frame}
}

\end{document}