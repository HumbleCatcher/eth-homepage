\documentclass[t,dvipsnames]{beamer}
\usepackage{cancel}
\usepackage{mathtools}
\graphicspath{{./assets/}}



\title{Diskrete Mathematik - Woche 8}
\author{Andreas Ellison}
\usetheme{Madrid}
\setbeamertemplate{navigation symbols}{}
\DeclareMathOperator{\isless}{less}

\DeclareMathOperator{\defn}{\stackrel{\text{def}}{\iff}}
\DeclareMathOperator{\defneq}{\stackrel{\text{def}}{\quad = \quad}}
\DeclareMathOperator{\idr}{id}
\newcommand{\isrel}[3]{#1 \, #3 \, #2}


\DeclareMathOperator{\lcm}{lcm}
\DeclareMathOperator{\ord}{ord}

\newcommand{\beamerblack}{
\setbeamercolor{frametitle}{fg=white}
\setbeamercolor{frametitle}{bg=black}
\setbeamercolor{background canvas}{bg=black}
\setbeamercolor{normal text}{fg=white}
\usebeamercolor[fg]{normal text}
\setbeamertemplate{footline}{}
\setbeamertemplate{itemize items}[circle]
\setbeamercolor{itemize item}{fg=white,bg=white}
\setbeamertemplate{enumerate items}[default]
\setbeamercolor{enumerate items}{fg=white,bg=white}
\setbeamercolor{itemize item}{fg=white,bg=white}
}

\newcounter{excounter}
\newcommand{\excount}{\stepcounter{excounter} \theexcounter{}}

\newcommand{\hintIllustration}{\textcolor{Magenta}{(Bild)}}


\begin{document}
{
\beamerblack
\begin{frame}
	\frametitle{Challenge!}
	Seien $a_1, a_2, \ldots, a_n \in \mathbb{Z}$.
	Gebe einen $\mathcal{O}(n^2)$ Algorithmus
	an, der eine zusammenhängende Teilfolge
	$a_i, a_{i + 1}, \ldots, a_j$ findet mit
	$1 \le i \le j \le n$, so dass
	$n \; \vert \; (a_i + a_{i + 1} + \ldots + a_j)$.

	Eine genaue Laufzeitanalyse vom angegebenem Algorithmus ist nicht
	nötig. \\~

	\textit{Tipp:} Betrachte die Summen
	$S_i \coloneqq \sum_{k=1}^{i} a_k$ für $1 \le i \le n$.
\end{frame}
}

{
\beamerblack
\begin{frame}
	\frametitle{Monoid-/Gruppeneigenschaften überprüfen}
	Für jede der folgenden Algebren, entscheide ob sie ein Monoid, eine Gruppe oder keines davon ist:
	\begin{itemize}
		\item
		      $\langle \mathbb{Z}; - \rangle$
		\item $\langle \mathcal{P}(\mathbb{N}); \cap \rangle$
	\end{itemize}
	\only<+(1)->{
		Das heisst, überprüfe
		\begin{enumerate}[1]
			\item \label{one} Ist die Operation assoziativ?
			\item \label{2} Gibt es ein neutrales Element?
		\end{enumerate}
		Falls 1 oder 2 scheitert, dann sind wir fertig. \\~

		\only<+(1)->{
			Sonst haben wir schon ein Monoid. Dann bleibt noch zu überprüfen
			\begin{enumerate}[3]
				\item Besitzt jedes Element ein Inverses?
			\end{enumerate}
		}
	}
\end{frame}
}

{
\beamerblack
\begin{frame}
	\frametitle{Subgroups}
	Let $\langle G; *, \hat{} \,, e \rangle$ be a group and define
	$$
		H = \{a \in G \mid \forall b \in G \; a * b = b * a\}
	$$
	(i.e. all commutative elements of $G$).
	Prove that $H$ is a subgroup of $G$. \\~

	\only<+(1)->{
		Das heisst, wir müssen zeigen für alle $a, b \in H$
		\begin{enumerate}[1]
			\item $e \in H$
			\item $a * b \in H$ (Operationen in $H$ bleiben auch in $H$, $H$ ist \textit{abgeschlossen} (\textit{closed}) bezüglich $*$)
			\item $\hat{a} \in H$ ($H$ abgeschlossen bezüglich $\hat{} \,$)
		\end{enumerate}
	}

\end{frame}
}

{
\beamerblack
\begin{frame}
	\frametitle{Mehr Subgroups}
	Let $H, H'$ subgroups of $G$. Prove
	$$
		H \cup H' \mbox{ is a subgroup of } G \implies H \subseteq H' \mbox{ or } H' \subseteq H.
	$$

	\only<+(1)->{
		\textit{Trick}: Aussagen der Form ``$S$ oder $T$'' kann man beweisen, indem man ``nicht $S \implies T$'' beweist.
	}
\end{frame}
}

{
\beamerblack
\begin{frame}
	\frametitle{Nützliche Tatsache}
	Beweise:
	$$
		g \mbox{ ist ein ``generator'' von }  \langle \mathbb{Z}_n; \oplus \rangle \iff \gcd(n, g) = 1.
	$$

	Folgendes könnte dabei hilfreich sein
	$$
		\lcm(a, b) \cdot \gcd(a, b) = a \cdot b.
	$$
\end{frame}
}


\begin{frame}
	\begin{Definition}[Ordnung]
		Sei $G$ eine Gruppe und $a \in G$. Dann ist $\ord(a)$ das kleinte $m \ge 1$, so dass $a^m = e$.
	\end{Definition}

	\only<+(1)->{
		\begin{Lemma}
			$\langle a \rangle = \{e, a, a^2, \ldots, a^{\ord(a) - 1}\}$ ist die kleinste Subgroup mit $a$.
		\end{Lemma}
	}

	\only<+(1)->{
		\begin{theorem}[Lagrange]
			Sei $H$ ein Subgroup von $G$. Dann gilt $|H| \mid |G|$.
		\end{theorem}
	}

	\only<+(1)->{
		\begin{Lemma}
			Für alle $a \in G$ gilt: $\ord(a) \mid G$.
		\end{Lemma}
	}

	\only<+(1)->{
		\begin{Lemma}
			Für alle $a \in G$ gilt: $a^{|G|} = e$.
		\end{Lemma}
	}
\end{frame}


\begin{frame}
	\frametitle{Next Level nützliche Tatsache}
	\begin{Lemma}
		Sei $g_1$ ein Generator von $\mathbb{Z}_m$ und $g_2$ ein Generator von $\mathbb{Z}_n$. Dann gilt
		$$
			\gcd(m, n) = 1 \iff (g_1, g_2) \mbox{ ein Generator von } \langle \mathbb{Z}_m; \oplus \rangle \times \langle \mathbb{Z}_n; \oplus \rangle.
		$$
	\end{Lemma}
\end{frame}

{
\beamerblack
\begin{frame}
	\frametitle{Beweisidee (eine Richtung)}
	Wir beweisen eine abgeschwächte Form davon:
	$$
		\gcd(m, n) = 1 \implies \langle (1, 1) \rangle = \langle \mathbb{Z}_m; \oplus \rangle \times \langle \mathbb{Z}_n; \oplus \rangle
	$$

\end{frame}
}

{
\beamerblack
\begin{frame}
	\frametitle{Wieder Subgroups}
	Liste alle Subgroups von $\langle \mathbb{Z}_3; \oplus \rangle \times \langle \mathbb{Z}_4; \oplus \rangle$ auf.

\end{frame}
}
\end{document}