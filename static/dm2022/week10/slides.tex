\documentclass[t,dvipsnames]{beamer}
\usepackage{cancel}
\usepackage{mathtools}
\graphicspath{{./assets/}}



\title{Diskrete Mathematik - Woche 10}
\author{Andreas Ellison}
\usetheme{Madrid}
\setbeamertemplate{navigation symbols}{}
\DeclareMathOperator{\isless}{less}

\DeclareMathOperator{\defn}{\stackrel{\text{def}}{\iff}}
\DeclareMathOperator{\defneq}{\stackrel{\text{def}}{\quad = \quad}}
\DeclareMathOperator{\idr}{id}
\newcommand{\isrel}[3]{#1 \, #3 \, #2}


\DeclareMathOperator{\lcm}{lcm}
\DeclareMathOperator{\ord}{ord}

\newcommand{\beamerblack}{
\setbeamercolor{frametitle}{fg=white}
\setbeamercolor{frametitle}{bg=black}
\setbeamercolor{background canvas}{bg=black}
\setbeamercolor{normal text}{fg=white}
\usebeamercolor[fg]{normal text}
\setbeamertemplate{footline}{}
\setbeamertemplate{itemize items}[circle]
\setbeamercolor{itemize item}{fg=white,bg=white}
\setbeamertemplate{enumerate items}[default]
\setbeamercolor{enumerate items}{fg=white,bg=white}
\setbeamercolor{itemize item}{fg=white,bg=white}
}

\newcounter{excounter}
\newcommand{\excount}{\stepcounter{excounter} \theexcounter{}}

\newcommand{\hintIllustration}{\textcolor{Magenta}{(Bild)}}
\newcommand{\domZ}{\mathbb{Z}}
\newcommand{\domQ}{\mathbb{Q}}
\newcommand{\domR}{\mathbb{R}}


\begin{document}

\begin{frame}
	\frametitle{Wieso Körper?}
	Mit Ringen haben wir etwas analoges zu $\mathbb{Z}$ oder $\mathbb{Z}_n$ gefunden. Wir können addieren, subtrahieren, multiplizieren und bekommen Eigenschaften wie
	\begin{itemize}[<+(1)->]
		\item $0a = 0$
		\item $(-1)b = -b$
		\item $(-a)(-b) = ab$
		\item $a + b = b \implies a = 0$
	\end{itemize}

	\only<+(1)->{Aber viele wichtige Eigenschaften und Operationen wie in $\mathbb{R}$ oder $\mathbb{Q}$ haben wir nicht:
		\begin{itemize}[<+(1)->]
			\item Wir können nicht dividieren (angenommen $a \neq 0$):
			      $$
				      ax = b \implies x = \frac{b}{a}
			      $$
			\item $a \cdot b = 0 \implies a = 0 $ oder $b = 0$ (gilt nur in einem ``integral domain'')
			\item $\deg(p(x) \cdot q(x)) = \deg(p(x)) + \deg(q(x))$
		\end{itemize}
	}
\end{frame}

\begin{frame}
	Daher:
	\begin{Definition}[5.26.]
		Ein Körper $F$ ist ein nichttrivialer kommutativer Ring mit $F^* = F \setminus \{0\}$. (Das heisst, jedes Element ausser 0 hat ein Inverses bezüglich Multiplikation.) \\~

		\only<+(1)->{
			Anders gesagt ist $\langle F \setminus \{0\}; \cdot, ^{-1}, 1 \rangle$ eine Gruppe.
		}
	\end{Definition}
	\only<+(1)->{
		Invertierbarkeit von $0$ verlangen wir nicht, aus dem gleichen Grund, wieso $\frac{1}{0}$ nicht definiert ist.
	}

\end{frame}

\begin{frame}
	\frametitle{Polynome über einem Körper}
	Die Polynome $F[x]$ haben viele Eigenschaften, die uns an $\mathbb{Z}$ erinnern.

	\only<+(1)->{
		In $\mathbb{Z}$ können wir für jedes $a, m \in \domZ$, $a$ als
		\begin{align*}
			 & a = m \cdot q + r, \quad r = R_m(a)
			\\ & \textcolor{Orange}{17 = 3 \cdot 5 + 2}
		\end{align*}
		schreiben.
	}

	\only<+(1)->{
		Genauso wie wir in $\mathbb{Z}$ über Reste sprechen können, können wir dies auch in $F[x]$ tun!
		\begin{Satz}[5.25.]
			Für jedes $a(x), m(x) \in F[x]$ gibt es $q(x), r(x)$, so dass
			$$
				a(x) = m(x) \cdot q(x) + r(x), \quad r(x) = R_{m(x)}(a(x))
			$$
			und
			$$
				\deg(r(x)) < \deg(m(x)) \quad \text{(analog zu $r < |m|$ in $\mathbb{Z}$)}
			$$
		\end{Satz}
	}
\end{frame}

{
\beamerblack
\begin{frame}
	\frametitle{Polynomdivision in $\domZ_p$}
	Teile $x^4 + 3x^2 + 4$ durch $4x^2 + 2x + 1$ in $\domZ_{5}$ mit Rest.
\end{frame}
}

\begin{frame}
	Mithile vom letzten Satz sehen wir, dass wir mit einem beliebigen Polynom $m(x)$ den Rest Modulo $m(x)$ betrachten können.
	\only<+(1)->{

		Analog also wie $\domZ_m$ können wir folgenden Ring definieren
		\begin{Definition}[5.35.]
			$$
				F[x]_{m(x)} = \{a(x) \in F[x] \mid \deg(a(x)) < \deg(m(x)) \}
			$$
			Das sind alle möglichen Reste Modulo $m(x)$.
		\end{Definition}
	}
\end{frame}

{

\beamerblack
\begin{frame}
	\frametitle{Modulo Arithmetik mit Polynomen}
	Berechne
	$$
		(x + 3)(x + 2)
	$$
	in $\domZ_5[x]_{x^2 + 4}$
\end{frame}

}

\begin{frame}
	\frametitle{Irreduzible Polynome}
	Genauso wie wir in $\domZ$ Primzahlen haben, haben wir in $F[x]$ \textit{irreduzible} Polynome.
	\only<+(1)->{
		\begin{Definition}[5.28.]
			Ein Polynom $a(x) \in F[x]$ heisst \textit{irreduzibel}, wenn es keinen Teiler $m(x)$ hat mit $0 < \deg(m(x)) < \deg(a(x))$ (analog dazu, dass eine Primzahl keine Teiler zwischen $1$ und $p$ hat).
		\end{Definition}
	}
	\only<+(1)->{
		Und analog wie in $\domZ$ kann auch jedes Polynom $a(x) \in F[x]$ in irreduzible Polynome faktorisiert werden.
	}
\end{frame}

\begin{frame}
	Folgende Tatsache, der uns von Polynomen über $\mathbb{R}$ bekannt ist, hilft uns, Irreduzibilität zu überprüfen.
	\only<+(1)->{
		\begin{lemma}[5.29.]
			$\alpha \in F$ ist eine Nullstelle von $a(x) \iff (x - \alpha) \mid a(x)$.
		\end{lemma}
	}
\end{frame}

{
\beamerblack
\begin{frame}
	\frametitle{Irreduzibilität überprüfen}
	Strategie um Teiler von $a(x)$ zu finden:
	\begin{enumerate}
		\item Überprüfe, ob $a(x)$ Nullstellen hat.
		\item Überprüfe für alle \textbf{irreduziblen} Polynome mit Grad $1 < d \le \deg(a(x))/2$, ob sie $a(x)$ teilen.
	\end{enumerate}

	\only<+(1)->{
		Sind die folgenden Polynome irreduzibel? Wenn nicht, dann faktorisiere sie
		\begin{itemize}[<+(1)->]
			\item $x^2 + 4$ in $\domZ_5$
			\item $x^3 + 2x^2 + 1$ in $\domZ_3$
			\item $x^4 + x^2 + 1$ in $\domZ_2$

			      \textit{Hinweis}: $x^2 + x + 1$ ist das einzige irreduzible Polynom in $\domZ_2$ von Grad 2.
		\end{itemize}
	}
\end{frame}
}

\begin{frame}
	\frametitle{Modulo und Teilbarkeit kombiniert}
	In $\domZ$ haben wir gesehen:
	\begin{lemma}[4.18.]
		$$
			ax \equiv_m 1
		$$
		ist lösbar für $x \in \domZ_m$ genau dann, wenn $\gcd(a, m) = 1$.
	\end{lemma}
	\only<+(1)->{
		Auch dazu haben wir auch ein Analogon:
		\begin{lemma}[5.36.]
			$$
				a(x)b(x) \equiv_{m(x)} 1
			$$
			hat eine Lösung $b(x) \in F[x]$ genau dann, wenn $\gcd(a(x), m(x)) = 1$.
		\end{lemma}
	}
\end{frame}

\begin{frame}
	Und wir können genauso von der multiplikativen Gruppe modulo $m(x)$ sprechen (vergleiche $\domZ^*_m$)
	\begin{Lemma}[5.36.]
		$$
			F[x]^*_{m(x)} = \left \{a(x) \in F[x]_{m(x)} \mid \gcd(a(x), m(x)) = 1 \right \}
		$$
	\end{Lemma}
\end{frame}

{
\beamerblack
\begin{frame}
	\frametitle{Der Ring $\domZ_p[x]_{m(x)}$}
	Betrachte den Ring $\domZ_3[x]_{x^2 + 2}$.
	\begin{enumerate}[<+(1)->]
		\item Bestimme alle Nullteiler von $\domZ_3[x]_{x^2 + 2}$.
		\item Liste alle Elemente der Gruppe $\domZ_3[x]^*_{x^2 + 2}$ auf.
		\item Bestimme das Inverse von $2x$ in der Gruppe $\domZ_3[x]^*_{x^2 + 2}$.
	\end{enumerate}

\end{frame}
}

{
\beamerblack
\begin{frame}
	\frametitle{``Verschachtelte'' Polynome}
	Ist das Polynom $xy^3 + xy^2 + (x + 1)y + x \in \domZ_2[x]_{x^2 + x + 1}[y]$ irreduzibel?
\end{frame}
}

\end{document}