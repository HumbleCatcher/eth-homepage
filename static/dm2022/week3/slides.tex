\documentclass[t,dvipsnames]{beamer}
\usepackage{cancel}

%Information to be included in the title page:
\title{Diskrete Mathematik - Woche 3}
\author{Andreas Ellison}
\usetheme{Madrid}


\DeclareMathOperator{\isless}{less}
\DeclareMathOperator{\isequal}{equal}
\DeclareMathOperator{\isprime}{prime}
\DeclareMathOperator{\isodd}{odd}
\DeclareMathOperator{\knows}{knows}
\DeclareMathOperator{\istypey}{hatEigenschaftY}

\begin{document}

\begin{frame}
	\frametitle{Prädikatenlogik - Alte Prüfungsfrage}
	\begin{enumerate}
		\item<1-> Finde eine Formel $F$, so dass in jeder Interpretation, die $F$ erfüllt, das Universum mindestens 2 Elemente hat. (Prüfung HS18)

			Tipp: ``$=$'', ``$\neq$'' sind erlaubt.

			\only<2->{\color{blue}
				Lösung:
				\begin{equation*}
					\forall x \exists y \, (x \neq y)
				\end{equation*}
				Beachte, dass das Universum \textit{nicht} leer sein darf. (Siehe ersten Satz von Kapitel 2.4.1 oder präziser Definition 6.34. im Skript).
			}
	\end{enumerate}
\end{frame}

\begin{frame}
	\frametitle{Prädikatenlogik - Interpretation finden}
	Finde für die folgenden Formeln Interpretationen, die sie erfüllen:
	\begin{enumerate}
		\item<1->
			\begin{equation*}
				\forall x \forall y \, \Big (E \big (f(x, y), f(y, x) \big ) \Big )
			\end{equation*}
			\only<2->{\color{blue}
				mögliche Lösung (Kommutativität): $U = \mathbb{Z}$, $E = \isequal$, $f(x, y) = x \cdot y$
			}
		\item<3->
			\begin{equation*}
				\forall x \forall y \forall z \, \Big (E \big (f(x, f(y, z)), f(f(x, y), z) \big ) \Big )
			\end{equation*}
			\only<4->{\color{blue}
				mögliche Lösung (Assoziativität): $U = \mathbb{R}$, $E = \isequal$, $f(x, y) = x + y$
			}
		\item<5->
			\begin{equation*}
				\exists G \forall F \, \Big ( C(F, G) \Big )
			\end{equation*}
			\only<6->{\color{blue}
				mögliche Lösung: $U =$ Menge aller Formeln, $C(F, G) = 1 \iff F \models G $
			}
	\end{enumerate}
\end{frame}

\begin{frame}
	\frametitle{Prädikatenlogik - Fehler finden}
	Welche der folgenden Formeln sind richtig für die Aussage
	``Every integer greater than 2 is a sum of two primes.'' und wenn nicht, wieso? Wir fixieren $U = \mathbb{Z}$.
	\begin{enumerate}
		\item<1->
			\begin{equation*}
				\forall x \exists y \exists z \, \Big ( 2 < x \rightarrow x = \isprime(y) + \isprime(z) \Big )
			\end{equation*}
			\only<2->{\color{red}
				Sehr falsch! ``$\isprime$'' ist ein Prädikat und gibt einen Wahrheitswert zurück. Wir können Wahrheitswerte nicht addieren.
			}
	\end{enumerate}
\end{frame}



\begin{frame}
	\frametitle{Prädikatenlogik - Fehler finden}
	Welche der folgenden Formeln sind richtig für die Aussage
	``Every integer greater than 2 is a sum of two primes.'' und wenn nicht, wieso? Wir fixieren $U = \mathbb{Z}$.
	\begin{enumerate}
		\setcounter{enumi}{2}
		\item<1->
			\only<1-1>{
				\begin{equation*}
					\forall x \exists y \exists z \, \Big (  2 < x \land \big ( \isprime(y) \land \isprime(z) \land x = y + z \big )  \Big )
				\end{equation*}
			}
			\only<2->{
				\begin{equation*}
					\bcancel{
						\forall x \exists y \exists z \, \Big (  2 < x \land \big ( \isprime(y) \land \isprime(z) \land x = y + z \big )  \Big )
					}
				\end{equation*}
			}
			\only<2->{\color{red}
				FAAALSCH!!!!! Richtig wäre
				\color{black}
				\begin{equation*}
					\forall x \exists y \exists z \, \Big (  2 < x \color{red} \rightarrow \color{black} \big ( \isprime(y) \land \isprime(z) \land x = y + z \big )  \Big )
				\end{equation*}
			}
			\only<3->{
				\color{red}
				Aussagen der Form ``Für jede x mit Eigenschaft Y gilt ...'' haben immer eine Formel der Form
				\begin{equation*}
					\forall x \, \big ( \istypey(x) \rightarrow \ldots  \big )
				\end{equation*}
			}
			\only<4->{
				Noch besser wäre
				\begin{equation*}
					\forall x  \, \bigg ( 2 < x \rightarrow
					\Big (\exists y \exists z \, \big ( \isprime(y) \land \isprime(z) \land x = y + z \big ) \Big ) \bigg ).
				\end{equation*}
			}
	\end{enumerate}
\end{frame}


\begin{frame}
	\frametitle{Prädikatenlogik - Logische Konsequenzen und Äquivalenzen}
	Sind die folgenden Aussagen wahr oder falsch? Wenn wahr, dann gebe eine (informelle) Begründung. Wenn falsch, dann gebe ein Gegenbeispiel.
	\begin{enumerate}
		\item<1->
			\begin{equation*}
				\forall x \forall y \, P(x, y)  \stackrel{?}{\equiv} \forall y \forall x \, P(x, y)
			\end{equation*}
			\only<2->{\color{ForestGreen}
				Wahr. Analogie: verschachtelte for loops.
			}
		\item<3->
			\begin{equation*}
				\neg \exists x \, P(x)  \stackrel{?}{\equiv} \forall x \, \neg P(x)
			\end{equation*}
			\only<4->{\color{ForestGreen}
				Wahr. ``Es gibt kein $x$, so dass $P(x)$ wahr ist'' ist das gleiche wie ``Für alle $x$ ist $P(x)$ falsch''.
			}
		\item<5->
			\begin{equation*}
				\exists x \, P(x) \land \exists x \, Q(x) \stackrel{?}{\models} \exists x \, \big ( P(x) \land Q(x) \big )
			\end{equation*}
			\only<6->{\color{red}
				Falsch. Gegenbeispiel: $U = \mathbb{Z}$, $P(x) = 1 \iff x > 0$, $Q(x) = 1 \iff y < 0$.
			}
	\end{enumerate}
\end{frame}

\begin{frame}
	\frametitle{Prädikatenlogik - Logische Konsequenzen und Äquivalenzen}
	Sind die folgenden Aussagen wahr oder falsch? Wenn wahr, dann gebe eine (informelle) Begründung. Wenn falsch, dann gebe ein Gegenbeispiel.
	\begin{enumerate}
		\item<1->
			\begin{equation*}
				\exists y \forall x \, P(x, y)  \stackrel{?}{\models} \forall x \exists y \, P(x, y)
			\end{equation*}
			\only<2->{\color{ForestGreen}
				Wahr. Nehme so ein $y'$ von links, welches mit allen $x$ zusammenpasst. Rechts können wir dann für jedes $x$ dieses $y'$ nehmen.
			}
		\item<3->
			\begin{equation*}
				\forall x \exists y \, P(x, y) \stackrel{?}{\models} \exists y \forall x \, P(x, y)
			\end{equation*}
			\only<4->{\color{red}
				Falsch. Gegenbeispiel: $U =$ reelle Zahlen ohne 0, $P(x, y) = 1 \iff x \cdot y = 1$
			}
		\item<5-> Seien $Q$, $Q'$ irgendwelche Quantoren (also $\forall$ oder $\exists$).
			\begin{equation*}
				Qx \, P(x) \equiv Q'x \, R(x) \stackrel{?}{\implies} Qx \, P(x) \models Q'x \, R(x)
			\end{equation*}
			\only<6->{\color{ForestGreen}
				Wahr. $F \equiv G \iff F \models G$ und $G \models F$.
			}
	\end{enumerate}
\end{frame}

\begin{frame}
	\frametitle{Prädikatenlogik - Die wichtigsten Regeln (für jetzt)}
	Für eine vollständigere Liste siehe Lemma 6.8. im Skript.
	\begin{enumerate}
		\item<2-> $Q x Q y \ldots \equiv Q y Q x \ldots (Q = \forall \mbox{ oder } Q = \exists)$
		\item<3-> Im Allgemeinen: $\forall x \exists y \ldots \not \equiv \exists y \forall x \ldots$
		\item<4->
			\begin{equation*}
				\forall x \, \left (P(x) \land Q(x) \right ) \equiv \forall x \, P(x) \land \forall x \, Q(x)
			\end{equation*}
			\begin{center}
				aber
			\end{center}
			\begin{equation*}
				\exists x \left ( P(x) \land Q(x) \right ) \not \equiv \exists x \, P(x) \land \exists x \, Q(x)
			\end{equation*}
		\item<5->
			\begin{equation*}
				\forall x \exists y \, (x > 0 \land (x = 2 \cdot y)) \equiv \forall x \, (x > 0 \land (\exists y \, x = 2 \cdot y))
			\end{equation*}
	\end{enumerate}
\end{frame}
\end{document}