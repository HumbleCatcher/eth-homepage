\documentclass[t,dvipsnames]{beamer}
\usepackage{cancel}
\usepackage{mathtools}
\graphicspath{{./assets/}}



\title{Diskrete Mathematik - Woche 11}
\author{Andreas Ellison}
\usetheme{Madrid}
\setbeamertemplate{navigation symbols}{}
\DeclareMathOperator{\isless}{less}

\DeclareMathOperator{\defn}{\stackrel{\text{def}}{\iff}}
\DeclareMathOperator{\defneq}{\stackrel{\text{def}}{\quad = \quad}}
\DeclareMathOperator{\idr}{id}
\newcommand{\isrel}[3]{#1 \, #3 \, #2}


\DeclareMathOperator{\lcm}{lcm}
\DeclareMathOperator{\ord}{ord}

\newcommand{\beamerblack}{
\setbeamercolor{frametitle}{fg=white}
\setbeamercolor{frametitle}{bg=black}
\setbeamercolor{background canvas}{bg=black}
\setbeamercolor{normal text}{fg=white}
\usebeamercolor[fg]{normal text}
\setbeamertemplate{footline}{}
\setbeamertemplate{itemize items}[circle]
\setbeamercolor{itemize item}{fg=white,bg=white}
\setbeamertemplate{enumerate items}[default]
\setbeamercolor{enumerate items}{fg=white,bg=white}
\setbeamercolor{itemize item}{fg=white,bg=white}
}

\newcounter{excounter}
\newcommand{\excount}{\stepcounter{excounter} \theexcounter{}}

\newcommand{\hintIllustration}{\textcolor{Magenta}{(Bild)}}
\newcommand{\deriv}[1]{\vdash_{#1}}


\begin{document}
{
\beamerblack
\begin{frame}
	\frametitle{Beweisaufgabe mit Polynomen}
	Let $F$ be a finite field. Show that there exists a non-constant polyomial $p(x) \in F[x]$ with no roots.
\end{frame}
}

{
\beamerblack
\begin{frame}
	\frametitle{Proof systems (HS21)}
	\includegraphics[width=\textwidth]{proof_systems.png}
	\begin{enumerate}[(a)]
		\item<+(1)-> Prove or disprove: If $\Pi$ complete, then $\Pi_1$ complete or $\Pi_2$ complete.
		\item<+(1)-> Prove or disprove: If $\Pi_1$ sound or $\Pi_2$ sound, then $\Pi$ sound.
	\end{enumerate}
\end{frame}
}

{
\beamerblack
\begin{frame}
	\frametitle{Kalküle (FS22)}
	Consider the calculus consisting of the following four derivation rules:
	\begin{align*}
		\varnothing                                   & \deriv{R_1} F \rightarrow F                    \\
		\{F\}                                         & \deriv{R_2} F \lor F                           \\
		\{\neg F \lor \neg F\}                        & \deriv{R_3} F \rightarrow (\neg F \lor \neg F) \\
		\{F \rightarrow (G \lor H), G \rightarrow H\} & \deriv{R_4} F \rightarrow H                    \\
	\end{align*}
	Formally derive $A \rightarrow \neg A$ from $\{\neg A \}$.
\end{frame}
}

\end{document}